\section{Inhalte}
Spektroskopische und mikroskopische Verfahren der Einzelmolekülfluoreszenz: Lokalisierung einzelner Moleküle, Tracking, Einzelmolekül-FRET, Fluoreszenzlöschung; Anwendungen von Einzelmolekülmethoden zur Untersuchung der Dynamik (z.B. Diffusion, Konformation, Bindungsstudien) einzelner Moleküle (z.B. Proteine, Nukleinsäuren, Liganden) in vitro und im zellulären Kontext; Methoden zur Überwindung der optischen Auflösungsgrenze in der Fluoreszenzmikroskopie (z.B. STED, STORM / PALM); Anwendung hochauflösender Fluoreszenzmikroskopie zur Untersuchung zellulärer Strukturen; quantitative, hochauflösende Fluoreszenzmikroskopie sowie gezielte Markierungsstrategien; Anwendung von Einzelmolekülmethoden zur Messung der Dynamik von Biomolekülen; Grundlagen der Fluoreszenz, der geometrischen Optik und des Aufbaus sowie der Funktionsweise von Mikroskopen