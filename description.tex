\section{Modulbeschreibung}
Ziel dieses Moduls ist, den Studierenden Einsichten in state-of-the-art Methoden der experimentellen Einzelmolekültechniken sowie in die hochauflösende Fluoreszenzmikroskopie zu geben. Es wird vermittelt, welche Fragestellungen wie beantwortet werden können, und wo die Grenzen bzw. Schwachpunkte der jeweiligen Methoden liegen. Der vermittelte methodische Hintergrund wird durch Beispiele aus der aktuellen Forschung ergänzt und vertieft.\\
Themen u.a. sind:
\begin{enumerate}
\item Fluoreszenz und Fluorophore
\item Grundlagen der Optik und Mikroskopie
\item Spektroskopische und mikroskopische Verfahren der Einzelmolekülfluoreszenz
\item Anwendungen von Einzelmolekülmethoden zur Untersuchung der Dynamik einzelner Biomoleküle
\item Methoden zur Überwindung der optischen Auflösungsgrenze in der Fluoreszenzmikroskopie
\end{enumerate}	

    